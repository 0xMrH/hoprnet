\subsection{Security Goals}
\label{sec:intro:securitygoals}

Generally, the HOPR protocol aims to hide the fact that two parties are communicating with each other, as well as the contents of the communication. This information should be hidden from any party external to the network, all intermediaries involved in transferring the data between the communicating parties, and even the communicating parties themselves, if one or both of them desires it.

Specifically, the HOPR protocol aims to build a network with the following features (definitions based on \cite{AnoA}, \cite{loopix} and \cite{sphinxpaper}) for senders, indicated by $A_n$ and recipients, indicated by $Z_n$.

\subsubsection{Sender anonymity}
In a network with sender anonymity, an observer is not able to tell if a particular packet was sent by any adversary-selected honest senders $A_1$ or $A_2$ to the honest recipient $Z$. Formally this is denoted as $\{A_1 \rightarrow Z, A_2 \not\rightarrow \}$ or $\{A_1 \not\rightarrow, A_2 \rightarrow Z \}$.

\subsubsection{Recipient anonymity}
In analogy to sender anonymity, here an observer is not able to tell if a particular packet was received by any adversary-selected honest recipients $Z_1$ or $Z_2$ from the honest sender $A$. Formally this is denoted as $\{A \rightarrow Z_1, \not\rightarrow Z_2 \}$ or $\{\not\rightarrow Z_1, A \rightarrow Z_2 \}$.

\subsubsection{Sender-recipient unlinkability}
For any pair of senders, $A_1$ and $A_2$, communicating with any pair of recipients, $Z_1$ and $Z_2$, an adversary must be unable to determine whether two packets travelled from $\{A_1 \rightarrow Z_1, A_2 \rightarrow Z_2 \}$ or $\{A_1 \rightarrow Z_2, A_2 \rightarrow Z_1 \}$.


These properties assume senders and recipients are honest, but they should hold for any honest senders and recipients of the adversary's choice.

To provide these features, the HOPR protocol builds on top of the Sphinx packet format \cite{sphinxpaper}, the specifics of which are explained in Section \ref{sec:sphinx}. HOPR also employs cover traffic, explained in Section \ref{sec:ct}. The additional parameters required for HOPR's incentive scheme are encapsulated in the Sphinx header as described in section \ref{sec:incentives:proofofrelay:challenge} as additional routing parameters and as such do not change the privacy guarantees of Sphinx.

