
\section{Introduction}
\label{sec:intro}

Internet privacy is a subset of data privacy and a fundamental human right as defined by the United Nations \cite{un2018}. In order to retain and exercise this right to privacy, each internet user must have full control of the transmission, storage, use, and disclosure of their personally identifiable information. However, the infrastructure and economics of our increasingly technologically driven world put great pressure on privacy, due to the various advantages (economic and otherwise) which knowledge of users' private information provides.

To combat this trend, multiple solutions are being developed to restore internet users' control of their private information. These solutions generally centre on providing truly anonymous communication, such that neither the content of the communication nor information about the communicators can be discovered by anyone, including the communicating parties.

HOPR is a decentralized incentivized mixnet \cite{mixnets} that employs privacy-by-design protocols. HOPR aims to protect users' transport-level metadata privacy, giving them the freedom to use online services safely and privately. HOPR leverages existing mechanisms such as the Sphinx packet format \cite{sphinxpaper} and packet mixing to achieve its privacy goals, but adds an innovative incentive framework to promote network growth and reliability. HOPR uses the Ethereum blockchain \cite{ethereum} to facilitate this incentive framework, specifically to perform probabilistic payments via payment channels.

\subimport{}{01-vision.tex}
\subimport{}{02-security-goals.tex}
\subimport{}{03-threat-model.tex}
\subimport{}{04-state-of-the-art.tex}
\subimport{}{05-assessment.tex}