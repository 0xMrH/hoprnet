\section{Cover Traffic}
\label{sec:CT}

Cover traffic (CT), also known as chaff, is randomly generated packets injected into a mixnet to increase its anonymity set. Since cover traffic is transported and mixed using the same procedures as `real' packet, cover traffic is indistinguishable from real traffic. Deploying cover traffic makes it more difficult for an attacker to conduct passive attacks against the network. This is a direct result of there being more traffic: passive attacks such as traffic analysis attacks become more expensive the more traffic there is to analyse.

Cover traffic also increases the anonymity level of the mix network by improving sender-recipient unlinkability. This is particularly important in the early stages of HOPR's lifecycle, since it can be reasonably expected that traffic through the network will initially be low. Without cover traffic, it would be easier for a global passive adversary to link the sender and recipient of a particular packet.

A drawback of cover traffic is that it increases bandwidth overhead, which ultimately reduces the capacity of the network. This trade-off between capacity, latency, and anonymity is a result of the ``anonymity trilemma" \cite{AnonymityTrilemma}. Since HOPR is a privacy network, it is reasonable to prioritise anonymity here. However, it is important to mitigate the deleterious effects of cover traffic as far as possible, particularly avoiding wasted transmissions. This is achieved through judicious choice of which nodes to open cover traffic channels to and select paths to, based on node importance, and closing covert traffic channels to nodes which are found to be offline. 

\subsection{Cover traffic nodes}
Cover traffic nodes (CT nodes) are HOPR nodes with a cover traffic service enabled, initially run by the HOPR Association (or third parties sponsored by the HOPR Association) for the purpose of generating cover traffic. In the future, anyone who fulfils the requirements for running a CT node in the HOPR network will be able to start one.

\subsection{Opening cover traffic channels} 
CT nodes must open and fund payment channels before they can send cover traffic packets via that counterparty. The number of channels a CT node can open is predefined and cannot be exceeded. Whenever payment channels are closed, the CT node replaces them by opening new channels, provided it still has funds. Channel opening is randomly weighted according to the importance of a node. A node's importance, $\Omega(N)$, is defined as follows:

$$\Omega(N) = st(N) * sum(w(C), \forall \; outgoingChannels(N))$$
where 
$$st(N) = uT(N) + tB(N)$$
and 
$$w(C) = \sqrt{(B(C) / st(Cs) * st(Cd))}$$
\\
where $N$ is the node, $w$ is the channel's weight, $st$ is the number of HOPR tokens staked by the node, and $C$ is a payment channel between two nodes in the HOPR network. For this channel, $Cs$ is the node which opened the payment channel, while $Cd$ is channel counterparty. $uT(N)$ is the balance of unreleased tokens for a node $N$. $B$ is the channel balance and $tB$ is the sum of the outgoing balances of each channel from node $N$.

 In accordance with this definition, a node's importance score increases with the channel balances of channels that node shares with other nodes with high total stake. This means that the odds of channel being opened to a particular node is not simply proportional to that node’s stake. Cover traffic packets are allocated in proportion to a node's importance rather than channel stake as a way to mitigate against selfish node operators seeking to maximize their own profits.

The way importance and weight are calculated prevents nodes from opening a few minimally funded channels to nodes with large stake and incentivizes nodes to stake in as many channels as possible,  including re-allocating their stake to their downstream nodes to receive cover traffic there. This discourages centralization and prevents ``thin traffic", which limits the size of the anonymity set and increases the risk of traceable packets, even with high per-hop latency.

Since distributing cover traffic comes at a cost to network bandwidth, it is important to try and minimise wasted transmission. Therefore, CT nodes distribute cover traffic based on node importance rather than node uptime. This is because individual nodes are liable to suddenly lose connectivity. Of course, this distribution method will still result in sending cover traffic to offline nodes. If a mix node is found to be offline, the channel to that node is closed and new one to a different node is opened. The offline node will have to wait until the CT node is ready to open a new channel to stand a chance of being reselected.

\subsection{Path selection and payout}
After opening channels, the CT node sends cover traffic packets at regular intervals. The path selection algorithm used for cover traffic is the same one used to select nodes for real traffic (see Section \ref{sec:path-selection} for more details) where nodes are selected at random but weighted by importance, starting at the CT node.

The first relay node is chosen with a probability proportional to the amount funded by the CT node, so every node has an equal chance of getting selected. For subsequent hops, nodes are selected with a probability weighted proportional to their importance. We use a weighted priority queue of potential paths to choose the next node. If the queue is empty then it fails.

As with real traffic, rewards are distributed to all selected nodes in a path as tickets which can be redeemed for HOPR tokens with a certain probability (see Section \ref{sec:tickets}). Thus there is a direct correlation between a node's importance and the cover traffic rewards it receives, ensuring that rewards are distributed fairly. These cover traffic rewards also serve as an incentive for node operators to open meaningful payment channels instead of e.g. Sibyls or nodes with insignificant stake who would thus not be able to relay a lot of traffic.


\subsection{Closing cover traffic channels}
To minimize wasted cover traffic, CT nodes will close any channel deemed likely to result in failed distribution and open new channel in its place. A cover traffic channel is closed and a new one opened if any of the following factors falls below its threshold:
  \begin{itemize}
      \item \textbf{Channel balance} The channel balance must remain higher than the minimum stake value, defined as:
      $$B(C) < L * \sigma$$
      where $B(C)$ is the channel balance, $L$ is the path length, and $\sigma$ is the ticket payout amount, which is currently hardcoded as 0.1 HOPR per ticket.
      \item \textbf{Connection quality} Connection quality, $q$, is defined as the fraction of `pings' the node has responded to within a 5 second cut-off. $q$ must not drop below the defined upper bound. 
      \item \textbf{Number of failed packets} The number of failed packets must stay above the failed packet threshold.
      $$ $$
  \end{itemize}
If a node is detected as having fallen below any of these thresholds, the CT node calls \textit{initiateChannelClosure()} and emits two events: \textit{ChannelUpdate} and \textit{ChannelClosureInitiated}.
