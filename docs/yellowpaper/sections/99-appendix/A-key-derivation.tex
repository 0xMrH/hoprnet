\section{Key Derivation}
\label{appendix:keyderivation}

During a protocol execution, a node derives a master secret $s_i$ from the SPHINX header that is then used to derive multiple sub-secrets for multiple purposes by using the BLAKE2s hash function within HKDF. HKDF is given by two algorithms: $\mathsf{extract}$ and $\mathsf{expand}$, where $\mathsf{extract}$ is used to extract the entropy from a given secret $s$, such as an elliptic-curve point, and produces the intermediate keying material (IKM). The IKM then serves as a master secret to feed $\mathsf{expand}$ in order to derive pseudorandom subkeys in the desired length.

\subsection{Extraction}

As a result of the packet creation and its transformation, the sender is able to derive a shared secret $s_i$ given as a compressed elliptic-curve point (33 bytes) with each of the nodes along the path.

$$s_i^{master} \longleftarrow \mathsf{extract}(s_i, 33, pubKey)$$

By adding their own compressed public key $pubKey$ as a salt, each node derives a unique $s_i^{master}$ for each $s_i$.

\subsection{Expansion}

Each subkey $s_i^{sub}$ is used for one purpose, such as keying the \textit{pseudorandomness generator} (PRG).

$$s_i^{sub} \longleftarrow \mathsf{expand}(s_i^{master}, length(purpose), hashKey(purpose))$$

\begin{center}
    \begin{tabular}{c | c|  c | c | l |}
        \cline{2-5}
                                             & Usage           & Name      & Length   & Hash Key (UTF-8)                \\
        \cline{2-5}
        \noalign{\smallskip}
        \cline{2-5}
        \multirow{4}{*}{\rotatebox{90}{\shortstack{SPHINX                                                               \\packet}}} & PRG             & $s^{prg}$ & 32       & \texttt{HASH\_KEY\_PRG}         \\
                                             & PRP             & $s^{prp}$ & 128 + 64 & \texttt{HASH\_KEY\_PRP}         \\
                                             & Packet tag      & $s^{tag}$ & 32       & \texttt{HASH\_KEY\_PACKET\_TAG} \\
                                             & Blinding        & $s^{bl}$  & 32 + 12  & \texttt{HASH\_KEY\_BLINDING}    \\
        \cline{2-5}
        \noalign{\smallskip}
        \cline{2-5}
        \multirow{2}{*}{\rotatebox{90}{PoR}} & acknowledgement & $s^{ack}$ & 32*      & \texttt{HASH\_KEY\_ACK\_KEY}    \\
                                             & ownKey          & $s^{own}$ & 32*      & \texttt{HASH\_KEY\_OWN\_KEY}    \\

        \cline{2-5}
    \end{tabular}
\end{center}

The values marked with a * are treated as field elements, hence there exists a non-zero probability that $\mathsf{expand}$ produces a value outside of the field. In this specific case, the utilized hash key is repeatedly padded by ``\_'' until $\mathsf{expand}$ returns a field element.
