\section{Threat Model}

Although we assume that nodes in the HOPR network can communicate reliably, the
network can still be damaged by malicious actors and node failures. We assume a
threat model with byzantine nodes with either the ability to observe all network
traffic and launch network attacks or inject, drop or delay messages. \\~\\There
are different attack vectors which could threaten the security of HOPR network,
in the following section we mention these attacks and the mitigation methods
used by HOPR to resist them:

\begin{itemize}
    \item \textbf{Sybil attacks:} An attacker uses a single node to forge
    multiple identities in the network, thereby bringing network redundancy and
    reducing system security. The attacker can de-anonymize the multi-hop packet traffic and thus links both sender and receiver's identities of the packet. The mitigation for sybil attacks is in the trust assumption of Sphinx packet format where only 1 honest relayer is needed and since the traffic is source routed, users can choose routes themselves to choose honest relayers. 
    \item \textbf{Eclipse attacks:} The attacker seeks to isolate and attack or
    manipulate a specific user that is part of the network. This is a common
    attack in peer-to-peer networks since nodes have a hard time identifying
    malicious ones as they don't have a global view of the whole network. The mitigation for this is by using one single honest entry node to DHT which will advertise only honest nodes via a smart contract DEADR (Decentralized Entry Advertisement and Distributed Relaying).
    
\end{itemize}
