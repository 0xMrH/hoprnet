\section{Threat Model}

Although we assume that nodes in the HOPR network can communicate reliably, the
network can still be damaged by malicious actors and node failures. We assume a
threat model with byzantine nodes with either the ability to observe all network
traffic and launch network attacks or inject, drop or delay messages. \\~\\There
are different attack vectors which could threaten the security of HOPR network,
in the following section we mention these attacks and the mitigation methods
used by HOPR to resist them:

\begin{itemize}
    \item \textbf{Sybil attacks:} An attacker uses a single node to forge
    multiple identities in the network, thereby bringing network redundancy and
    reducing system security. This attack is expensive to conduct in practice since
    the attacker must stake lots of HOPR tokens within each malicious node they
    create in order to increase their probability of being chosen as a relayer and
    thus attacking the network.
    \item \textbf{Eclipse attacks:} The attacker seeks to isolate and attack or
    manipulate a specific user that is part of the network. This is a common
    attack in peer-to-peer networks since nodes have a hard time identifying
    malicious ones as they don't have a global view of the whole network. The cost
    of launching an eclipse attack is high since HOPR nodes are constantly
    challenging other peers and keeping a reputation score for each node.
    \item \textbf{Camouflage attacks:} A malicious node pretends to be an honest
    one most of the time. When its reputation value reaches a high level, it
    occasionally attacks the system. Since the attacker needs a long time to gain
    enough reputation score and be selected. Based on this, the system can still
    perform well.
    
\end{itemize}
