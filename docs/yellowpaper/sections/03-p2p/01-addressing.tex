\subsection{Addressing}
\label{sec:p2p:addressing}

Each node possesses an ECDSA public key which is used to derive unique identifier in the network, called \textit{HOPR address}. The HOPR address uses the self-describing multiformats\footnote{\href{https://multiformats.io}{https://multiformats.io}} standard which creates a prefix how the address is supposed to be interpreted. For ECDSA public keys on the secp256k1\footnote{\href{https://en.bitcoin.it/wiki/Secp256k1}{https://en.bitcoin.it/wiki/Secp256k1}} curve, this yields the binary multiformats prefix $0x002508021221$\footnote{$0x002508021221_{16} = [ 0_{10}, 37_{10}, 8_{10}, 2_{10}, 18_{10}, 33_{10} ]$, meaning 37 bytes, containing a compressed ECDSA public key on the secp256k1 curve and the key itself consists of 33 bytes}. The string representation is the Base58\footnote{\href{https://en.bitcoin.it/wiki/Base58Check_encoding}{https://en.bitcoin.it/wiki/Base58Check\_encoding}}-encoding of the bitstring.

\begin{center}
    $id = base58.encode(0x002508021221 \ || \ pubKey)$
\end{center}

Identifiers distinguish nodes from each other, whereas addresses allow nodes to establish a connection to each other. HOPR addresses follow the self-describing \textit{multiaddr} standard. There are two types of addresses: \textit{direct addresses} and \textit{relay addresses}.

\paragraph{Direct Addresses}

Nodes that are able to gain control over a TCP socket can have one or more direct address which are determined by their network adapter. HOPR supports running multiple nodes on the same machine, multiple nodes in the same local network and nodes running behind public IP addresses. Following the \textit{multiaddr} standard, an address includes the IP address family, e.g. IPv4, the actual host address, the transport protocol and the utilized port, followed by the HOPR address. The string representation is given by

\begin{center}
    $addr = \mathsf{/ip4/1.2.3.4/tcp/9091/}\text{\textless$id$\textgreater{}}$
\end{center}

\paragraph{Relay Addresses}

HOPR nodes can also reach each other indirectly by asking other nodes in the network to forward their traffic to the desired destination. This is especially necessary if nodes are operating behind a NAT or a restrictive firewall. Using a relay address means first establishing a connection to the relay and asking the relay node to extend the connection to the final destination. If the relay node manages to reach the final destination, it opens a general-purpose connection that allows node to exchange data, such as mixnet packets or status messages. Following the \textit{multiaddr} standard, the relay address starts with the HOPR address of the relay node, that is reached using a p2p connection, which attempts to establish a circuit by using a p2p connection to the destination.

\begin{center}
    \textsf{/p2p/}\textless\textsf{$relayId$}\textgreater\textsf{/p2p-circuit/p2p/}\textless{}\textsf{$id$}\textgreater{}
\end{center}

To prevent congestion, relay nodes come with a limited number of relay slots and nodes need to register with the relay nodes to use them as relays.

At startup, each node fetches all nodes that have announced themselves on-chain with a routable address. From this list, each node selects those five relay nodes with the lowest latency and also have a free relay slot. Afterwards, the node creates for each relay node a relay address and publishes them within the DHT and keeps the connection to the relay node open which preserves the address mapping of the NAT.
