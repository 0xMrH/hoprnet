\subsection{Availability Measurement}
\label{sec:p2p:availability}

When learning about a new node, it remains unclear to the initiator of the connection whether this node is available or not. Since nodes might temporariliy or permanently leave the network, this also applies to already known nodes. Sending a mixnet packet requires a randomly sampled path through the network, it becomes necessary to estimate a nodes' availability and to provide a list of nodes the are expected to be online. Note that nodes cannot ping nodes just before they use them as a relay node because this would reveal the chosen path, hence nodes need to continuously and independently grow their view of availability of other nodes.

Within the HOPR network, this is achieved by using a heartbeat mechanism that continuously sends ping requests to other nodes. A node is considered online if it replied to a \textit{PING} attempt with a correct \textit{PONG} response or if the node has recently received a \textit{PING} request by that node.

Each node counts successful and unsuccessful \textit{PING} attempts and uses this information to create a health score for other nodes. To prevent from continuously probing offline nodes, the heartbeat mechanism increases the probing interval in case the node was not reachable.

\subsubsection{Ping mechanism}
\label{sec:p2p:ping-mechanism}

Ping messages are small messages that are sent in order to see if the destination is available. Each request should be unique in order to be able to clearly distinguish it from potential retransmissions, hence each ping request contains a random 16-byte string. To make sure that the destination has indeed processed the ping request, ping responses, ``\textit{PONG}'' messages, are considered valid if they include the SHA256 hash of $r$.

\subsubsection{Health Score}
\label{sec:p2p:health-score}

While running, each node maintains a list of ``interesting nodes'' from which it keeps measuring the availability. At startup, this list is initiated with all nodes to which the node has funded outgoing payment channel, see section describing the \lcnameref{sec:incentives} for further details. Nodes on the list start with a initial health score of $0.2$. Whenever the node successfully pings a node on the list or the node passively gets pinged by a node, their health score increases by $0.1$ up to a maximum of $1.0$. Analogously, each unsuccessful attempt decreases the health score by $0.1$ down to $0$ which means that the node is considered offline.

Nodes with a health score greater than $0.5$ are expected to have a sufficiently high availability and can be used as a relay when sampling a mixnet path.

\subsubsection{Exponential Backoff}
\label{sec:p2p:exponential-backoff}

To provide a dynamic trade-off for both cases, HOPR utilizes a heartbeat with exponential backoff, the time until the next \textit{ping} is sent to a node increases with the number of failed ping attempts $n_{fail}$ since the last sucessful attempt or the network start. A successful response to \textit{ping} resets the backoff timer.

$$ t_{bo} = {t_{base}}^{{f_{bo}}^{n_{fail}}} $$

where $t_{bo}$ is the backoff time, $t_{base} = 2s$ is the initial backoff time and $f_{bo} = 1.5$.

The maximal backoff time of 512 seconds corresponds to $n_{fail} = 5$.

