\subsection{Ticket Redemption}
\label{sec:tickets:redemption}

After running through the validations of the previous section, the node $n$ ends up with a set of stored tickets which it considers to be a win, hence

$$ stored := \{ t \in Tickets \ | \ isWinner(t) \land t.recipient = n \}$$

Each ticket $t$ is given a  \lcnameref{sec:tickets:issuance:ticketindex}, which means tickets need to be redeemed in order which is why the node first creates an ordered set $ordered$ out of the set $tickets$ and proceeds with the first ticket.

Redemption means that the node now proves for each $t \in stored$ one-by-one to the smart contract that $t$ is indeed a win. If successful, the smart contract transfers the stated incentives to the account of the node, see paragraph \lcnameref{sec:tickets:redemption:assettransfer}.

In contrast to ticket recipients, the smart contract considers a ticket only valid if the redeemer is able to provide a $response$ that solves $ticket.challenge$ and a value $opening$ that opens the most recent $commitment$ stored on-chain. Note that the smart contract thereby acts as a trusted third party that forces the node reveal additional cryptographic material despite the signature of the ticket is valid. This is possible because the blockchain consensus makes it infeasible to add state changes which have not been the result of a method execution in the smart contract.

\paragraph{Challenge}
\label{sec:tickets:redemption:challenge}

Solving a challenge $C$ means finding a value $r \in \mathbb{F}$ such that $r \cdot G =C$. Hence, in order to check this equation, the smart contract needs to compute a scalar multiplication of an elliptic curve point, which is as of writing of the paper not directly available within Ethereum.

Instead, Ethereun allows to efficiently implement a function $mul'$:

$$ mul': x \in \mathbb{F} \mapsto ethAddr (x \cdot G)$$

where $ethAddr: \{0,1\}^{64} \mapsto \{0,1\}^{20}$ maps uncompressed elliptic curve points to Ethereum addresses. Hence, the smart contract compares the computed Ethereum address against the challenge stated in the ticket.

\paragraph{Issuer signature}
\label{sec:tickets:redemption:signature}

By computing $C' = mul'(response)$ the smart contract is able to recompute the hash of the ticket as

\begin{multline*}
      ticketHash = keccak256 (recipient \ || \ C' \ || \ ticketEpoch \ || \ amount \ || \\
      invWinProb \ || \ index \ || \ channelEpoch)
\end{multline*}

and is thus able by using the provided signature to recover the public key of the ticket issuer. By now having both Ethereum addresses, the one of the issuer and the one of the recipient, the smart contract is able to compute the identifier $channelId$ of the utilized payment channel.

\paragraph{Payment channel validation}
\label{sec:tickets:redemption:channel}

As the previous steps were computed without any feedback to the computed values, the computed $channelId$ will either lead to a non-existing entry in case there is no such channel known to the blockchain, or to a record of a payment channel. Due to the usage of a collision-resistant hash function, it is assumed to be infeasible for an attacker to find a second pre-image that maps the ticket hash to a specific $channelId$. Hence, either the challenge is correct and there is a payment channel or any of the previous conditions is not met.

If $channelId$ leads to a payment channel record, the smart contract checks that $channel.state = OPEN$  and $channel.amount \le ticket.amount$ and rejects the ticket otherwise.

\paragraph{Replay protections}
\label{sec:tickets:redemption:replay}

A ticket can be valid due to valid signature and correct $response$, but as it controls an asset transfer and thereby initiates on-chain state changes, it must be valid exactly \textit{once}.

Each ticket is given an ongoing serial number $i$ and each ticket redemption sets the on-chain value $channel.index$ to $ticket.index$ if $ticket.index > channel.index$. Otherwise, the ticket is rejected.

Analogously, each reincarnation of the payment channel, i.e. a sequence of $open$ and $close$, increases the channel epoch counter. To turn tickets issued for previous incarnations of the payment channel invalid, the smart contract rejects all tickets with $ticket.channelEpoch \neq channel.epoch$

Last but not least, each renewal of the on-chain commitment increases the ticket epoch counter. To prevent from the ticket issuers from ticket recipients resetting the ticket epoch counter to values that they find more beneficial, e.g. in order to tweak the ticket's winning probability and turn previously losing tickets into winning ones, the smart contract rejects all tickets for which $ticket.ticketEpoch \neq channel.ticketEpoch$.

\paragraph{Commitment}

By opening a payment channel from ticket issuer to ticket recipient, the ticket recipient needs to store a series of commitments in the smart contract. Let $commitment$ be the most recent one. By submitting a ticket, the ticket recipient recipient peals off one of the previously deposited commitments, hence need to check that $open$ is a valid opening for $commitment$. This done by checking if

$$ channel.commitment = keccak256 (opening)$$

If false, the smart contract rejects the ticket. See section \ref{sec:incentives:commitment} for further details on the commitment scheme.

\paragraph{Ticket luck}
\label{sec:tickets:redemption:ticketluck}

Redeeming a ticket includes two sources of entropy of entropy: $response$ to the stated $challenge$ that is known by the ticket issuer, and $open$ that opens the most recent $commitment$ and is solely known to the ticket recipient. By submitting both values to the smart contract, the smart contract is able to determine whether ticket is a winner. It checks

$$ \mathsf{keccak256} ( \ \mathsf{keccak256}(ticketData) \ || \ solution \ || \ opening \ ) < ticket.winProb $$

and causes a revert otherwise.

\paragraph{Asset transfer}
\label{sec:tickets:redemption:assettransfer}

If none of the previous checks have failed, the smart contract transfers the included tokens in the ticket to the recipient. This can happen in two ways: either there is an open payment channel in the other direction, namely from ticket recipient to ticket issuer and $ticket.amount$ is credited to that payment channel or the tokens are directly transferred to the recipient's account.
