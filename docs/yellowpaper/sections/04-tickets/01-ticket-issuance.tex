\subsection{Ticket Issuance}
\label{sec:tickets:issuance}

Before a node is able to issue tickets for another node, it needs to lock funds on-chain to cover for the current as well as future tickets. Locking funds is considered equal to staking tokens in the HOPR network as it allows the node to create mixnet packets and act as relayer. By locking tokens in the smart contract, the node creates a unidirectional payment channel towards the recipient and is thus able to convince the recipient that it is eligible to issue tickets.

As ticket issuance happens without any interaction to the blockchain, it is the duty of the node who receives the ticket to check whether there were any tokens locked on-chain and to keep track about previously issued tickets. If there is no record on-chain about any locked funds or if the sum of the received tickets exceed the amount of tokens that were locked on-chain, the recipient should refuse the ticket.

Tickets are sent together with a mixnet packet and include the incentive for processing and forwarding the packet to the next downstream node. To meet the \lcnameref{sec:intro:securitygoals}, neither issuance nor redemption must be linkable to the creation or processing of mixnet packets. Therefore, each ticket is given a winning probability, which means that not every issued ticket leads to a claimable incentive and, moreover, those tickets who turn out to be a winner cover the missed incentives of losing tickets. It turned out that is not only beneficial in terms of privacy but also helps to keep transaction costs resulting from on-chain interactions minimal.

When issuing a ticket, the issuer picks a winning probability $winProb > 0$ and starts creating the data structure $ticketData$ by setting the intended value $value$ through

$$ ticketData.value := \frac{value(ticket)}{winProb} $$

Note that the ticket issuer should not choose $winProb$ too low as the recipient might refuse the ticket due to inappropriate winning probability.

\begin{figure}[H]
      \centering
      \begin{tabular}{c|l|c|c|}
            \cline{2-4}
                                                        & \textbf{Value}                                    & \textbf{Ethereum datatype} & \textbf{size (in bytes)} \\
            \cline{2-4}
            \noalign{\smallskip}
            \cline{2-4}
            \multirow{7}{*}{\rotatebox{90}{TicketData}} & \nameref{sec:tickets:issuance:recipient}          & address                    & 20 bytes                 \\
                                                        & \nameref{sec:tickets:issuance:challenge}          & bytes32                    & 32 bytes                 \\
                                                        & \nameref{sec:tickets:issuance:ticketepoch}        & uint256                    & 32 bytes                 \\
                                                        & \nameref{sec:tickets:issuance:ticketvalue}        & uint256                    & 32 bytes                 \\
                                                        & \nameref{sec:tickets:issuance:winningprobability} & uint256                    & 32 bytes                 \\
                                                        & \nameref{sec:tickets:issuance:ticketindex}        & uint256                    & 32 bytes                 \\
                                                        & \nameref{sec:tickets:issuance:channelepoch}       & uint256                    & 32 bytes                 \\
            \cline{2-4}
            \noalign{\smallskip}

            \cline{2-4}
            \multirow{3}{*}{\rotatebox{90}{Sig}}        & Signature $r$                                     & bytes32                    & 32 bytes                 \\
                                                        & Signature $s$                                     & bytes32                    & 32 bytes                 \\
                                                        & Recovery value $v$                                & uint8                      & 1 byte                   \\
            \cline{2-4}
      \end{tabular}
      \caption{Structure of a ticket.}
      \label{fig:ticketdata}
\end{figure}

\paragraph{Recipient}
\label{sec:tickets:issuance:recipient}

The Ethereum address of the recipient, derived from the recipient's public key. This nails down the ticket to one specific payment channel, the one from ticket issuer to ticket recipient. Note that Ethereum addresses are computed as

$$ ethAddr: pubKey \in \{ 0,1 \}^{64} \mapsto \mathsf{keccak256}( pubKey).\mathsf{slice}(12,32)$$

(the last 20 bytes of the keccak256 hash of the uncompressed ECDSA public key).

\paragraph{Challenge}
\label{sec:tickets:issuance:challenge}

Tickets are issued locked, hence the embedded incentive is not yet claimable by the ticket recipient but their validity can be verified. Locked means that the ticket states a challenge which needs to be solved before being able to claim the embedded incentive. This mechanism servers as a building block for \lcnameref{sec:incentives:proofofrelay}.

\paragraph{Ticket epoch}
\label{sec:tickets:issuance:ticketepoch}

Ticket redemptions relies on providing the value $opening$ to a series of commitments that have previously been stored on-chain by the ticket recipient. To make sure that the party who wants to redeem tickets, is always able to compute the opening to a commitment, there is the opportunity to renew the on-chain commitment. As this allows the ticket recipient to change the entropy that is used to determine whether a ticket is a winner, the smart contract stores a counter that increases on every renewal and the ticket issuer signs the current value. This makes sure, that each commitment renewal invalidates all previously issued but not yet redeemed tickets.

\paragraph{Ticket value}
\label{sec:tickets:issuance:ticketvalue}

The ticket value is given by the intended $value$ divided by the winning probability $winProb$ in the base unit of the token, which is $10^{-8}$. Hence, sending $1$ HOPR with a winning probability of $1$ leads to $ticket.value = 10^8$.

\paragraph{Winning probablity}
\label{sec:tickets:issuance:winningprobability}

The proportion of tickets which lead to an actual payout is determined by their winning probability. To prevent from issues resulting from roundings, $ticketData$ includes the inverse winning probabilty that is normalized with the common base of Ethereum, which is $2^{256} - 1$. Hence,

$$ ticketData.invWinProb := winProb * (2^{256} -1)$$

\paragraph{Ticket index}
\label{sec:tickets:issuance:ticketindex}

Each ticket is labeled by an ongoing serial number named ticket index $i$ and its current value is stored in the smart contract. Whenever a ticket gets redeemed, the stored value is updated to the value given by the redeemed ticket and thus invalidates all tickets with index $i' \le i$. This is necessary to have each ticket being valid exactly once. Since ticket issuance does not change the value stored in the smart contract and tickets and tickets with unchanged ticket index are worthless, it is the duty of the ticket recipient to make sure that ticket index keeps increasing.

\paragraph{Channel epoch}
\label{sec:tickets:issuance:channelepoch}

Payment channels can get opened and closed as often as their participants like to, see section \ref{sec:incentives:channels} for more information. To make sure that tickets from previous channel incarnations lose their value once the channel is reopened, tickets include the current channel epoch counter and the smart contract considers the ticket invalid if the signed channel epoch does not match the stored channel epoch.