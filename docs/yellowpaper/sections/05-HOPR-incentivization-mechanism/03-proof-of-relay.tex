\subsection{Proof of Relay}
\label{sec:proofofrelay}

HOPR incentivizes packet transformation and delivery using a mechanism called
\textbf{proof of relay}. This mechanism guarantees that a node's relay services are
verifiable.

\paragraph{Construction}

\begin{itemize}
    \item Every packet is sent together with a ticket.
    \item Each ticket contains a challenge.
    \item The validity of a ticket can only be checked on reception of the
        packet but the on-chain logic enforces a solution to the challenge
        stated in the ticket.

\end{itemize}

\subsubsection{Challenge}

$A$ creates a shared group element $\alpha_i$ with all the relay nodes in the channel (B-C-D-Z) by using an offline version of the Diffie-Hellman key exchange. This shared key is a session key generated from the master DH Sphinx key and will be used to derive new keys $s_i^{(0)}$ and $s_{i}^{(1)}$. 
The first key, $s_i^{(0)}$, will be used as the node's \textit{own key share} and the second key, $s_i^{(1)}$, will be used as the \textit{acknowledgement key}. The acknowledgement key will be embedded in the acknowledgement for a packet and thereby unlocks the reward for the previous relayer, earned for transforming and delivering the packet.
    \paragraph{Key derivation}
    $A$ first creates a ``session secret", $s_i$, then uses it as a seed to derive subkeys $s_i^{(0)},s_i^{(1)}$ for each node on the route. This is done using an HKDF (HMAC-based Key Derivation Function), a cryptographic hash function that derives one or more secret keys from a secret value using a pseudorandom function. The key derivation works as follows:
    \begin{itemize}
        \item \textbf{Extract}: Creates a pseudo-random key $s_i$   
        $$HKDF.extract(h_b, |h_b|, (\alpha_i* privKey) || pubKey)$$   

        \item \textbf{Expand}: This step takes the output of the previous one, $s_i$, as a seed and creates output key material $s_i^{(0)},s_i^{(1)}$ which is expanded from hashes of $s_i$ and an optional info message (salt). The process occurs as follows:
        $$HKDF.expand(h_b, |h_b|,s_i, |s_i|, hashKey)$$
    \end{itemize}
    where $||$ is concatenation and $*$ is scalar multiplication on the curve. $h_b$ is the BLAKE2s 256 hash algorithm, $|h_b|$ is its length, $s_i$ is the pseudo random key used as a seed and $|s_i|$ is its length. $hashKey$ is an identifier used to create a``virtual" hash function for each purpose, e.g., one for the PRG, one for PRP, etc. and also for the proof-of-relay scheme. If the result of HKDF does not lead to a field element, $hashKey$ is padded until it does.
    \\~\\ $A$ provides a hint to the expected value $s_{i+1}^{(1)}$ that a node $n_i$ is expected to get from the next downstream node.
    The hint value, $H$, is computed as 
    \begin{align}  
        H_i&=s_{i+1}^{(1)}*G,
         \end{align}
    where $*$ is the curve multiplication operation and $G$ is a generator of the curve (the same used in the \ref{sec:sphinx} section). 
    \newline The hint for party $n_i$ is used to check whether the returned value $s_{i+1}'^{(1)}$ matches the promised value $s_{i+1}^{(1)}$ by checking whether $H_i$ equals $s_{i+1}'^{(1)}*G$. 
   \\The sender $A$ also creates a challenge $T_{c_i}$, such that 
   \begin{align}  
    T_{c_i}&=(s_i^{(0)}+s_{i+1}^{(1)})*G
     \end{align}
   Since proof of relay is used to ensure the relay services of nodes are verifiable, it is the duty of each node to check that given challenges are derivable from the given and expected information.
Packets with inappropriate challenges should be dropped, as they might not result in winning tickets with the expected probability.
\\~\\The values $H_i$ and $T_{c_i}$ are sent with the routing information $\beta_i$ as follows:
$$\beta_i=y_{i+1}\|H_i\|T_{c_i}\|\gamma_{i+1}\|\beta_{{i+1}_{[ \,0....(2r-1)\kappa-1\,] }}\oplus \rho(h_{\rho}(s_{i}))_{[ \,0....(2r+1)\kappa-1\,]}$$
By decrypting $\beta_i$, each mix node $n_i$ will retrieve the public key of the next downstream node and both the hint and challenge required by proof of relay to ensure relay services are verifiable.

\begin{comment}
 \begin{figure}[H]
    \centering
    \begin{tabular}{| m{2em} | m{15em} | m{2em} |}
        \hline
        $\alpha$ & $\beta$                   & $\gamma$ \\
                 & \begin{tabular}{| c m{2em} | m{3em} | m{6em} |}
            \hline
            \multicolumn{2}{| c |}{$Y_B$} & $hint_B$                 & $challenge_{BC}$ \\
            \hline
            \multicolumn{2}{| c |}{$Y_C$} & $hint_C$                 & $random$         \\
            \hline
            \multicolumn{2}{| c |}{$Y_D$} & $hint_D$                 & $random$         \\
            \hline
            End                           & \multicolumn{3}{| l |}{}                    \\
            \hline
        \end{tabular} &          \\[3em]
        \hline
    \end{tabular}
    \caption{Sphinx with PoR}
    \label{fig:Sphinx with PoR}
\end{figure}
\end{comment}
