\subsection{On-chain Commitment}
\label{sec:onchaincommitment}

HOPR uses a commitment scheme to deposit values on-chain and reveal them once a node redeems an incentive for relaying packets. This comes with the benefit that the redeeming party must disclose a secret that is unknown to the issuer of the incentive until it is claimed on-chain. The $\textbf{opening}$ $\mathsf{Open}$ and the $\textbf{response}$ $r$ to the proof-of-relay challenge (called \textit{nextCommitment} and \textit{proofOfRelaySecret} in the smart contract) are then used to prove to the smart contract that the node has a legitimate claim to the funds. HOPR uses a computationally hiding and binding commitment scheme:

\begin{defnsub}
    % Currently leaving out further details such as unconditionally/computationally binding / hiding
    A commitment scheme $\mathsf{C_m} = (\mathsf{Commit}, \mathsf{Open})$ is a protocol between two parties, $A$ and $B$, that gives $A$ the opportunity to store a value $comm = \mathsf{Commit}(x)$ at $B$. The value $x$ stays unknown to $B$ until $A$ decides to reveal it to $B$.
    For any $m\in \mathbb{M}$, $(c,d)\leftarrow \mathsf{Commit(m)}$ is the commitment/opening pair for $m$ where $c = c(m)$ serves as the commitment value, and $d = d(m)$ as the opening value.
    
 \noindent\textbf{Hiding:} A commitment scheme is called \textbf{computationally hiding} if the following holds:
    $$\forall \;x\neq x' \; \{\mathsf{Commit}_k(x,U_k)\}_{k\in\mathbb{N}}\approx\{{\mathsf{Commit}_k(x',U_k)}\}_{k\in\mathbb{N}}.$$ This means both probability ensembles are computationally indistinguishable, such that $U_{k}$ is the uniform distribution over the $2^{k}$ opening values for the security parameter $k$.
    
\noindent\textbf{Binding:} A commitment scheme is called \textbf{computationally binding} if for all bounded polynomial adversary \textit{Adv} algorithms that run in time $t$ with output $x,x',open,open'$, the following holds:
    $$P	[ \,x\neq x' \; and \; \mathsf{Commit}(x,open)={\mathsf{Commit}(x',open')}] \,\leq \epsilon$$ A computationally bounded adversary has limitations on their computational resources. 

\end{defnsub}

\subsubsection{Commitment phase}

Once a node is the destination of a HOPR unidirectional channel, it generates a master key $comm_0$ randomly and uses it to create an iterated commitment $comm_i$ such that for every $i \in \mathbb{N}_0$ and $i > 0$ it holds that $$ \mathsf{Open}(comm_{i}, comm_{i-1}) = \top,$$ which means opening $comm_{i}$ with $comm_{i-1}$ holds true.

The iterated commitment is computed as $$comm_n = h^n(comm_0),$$ where $h$ is a pre-image resistant hash function (we use the keccak256 hash function, which is also used in Ethereum). The master key should be pseudorandom, such that all intermediate commitments $comm_{i}$ for $i \in \mathbb{N}_0$ and $0 < i \l en$ are indistinguishable for the ticket issuer from random numbers of the same length. This is necessary to ensure that the ticket issuer is unable to determine whether a ticket is a winner or not when issuing the ticket. This makes it infeasible for the ticket issuer to tweak the challenge to such that it cannot be a winner.

When dispatching a transaction that opens the payment channel, the commitment $comm_n$ is stored in the channel structure in the smart contract and the smart contract will force the ticket recipient to reveal $comm_{n-1}$ when redeeming a ticket issued in this channel. The number of iterations $n$ can be chosen as a constant and should reflect the number of tickets a node intends to redeem within a channel.

\subsubsection{Opening phase}

In order to redeem a ticket, a node must reveal the opening to the current commitment $comm_i$ that is stored in the smart contract for the channel. Since the opening $comm_{i-1}$ allows the ticket issuer to determine whether a ticket is going to be a winner, the ticket recipient should keep $comm_{i-1}$ until it is used to redeem a ticket. Tickets lead to a win if: $$\mathsf{h}( t_h, r_i, comm_{i-1} ) < P_w,$$ where $P_w$ is the ticket's winning probability and $$t_h=\mathsf{h}(t) \;and\; \mathsf{Open}(comm_i, comm_{i-1}) = \top.$$ Since $comm_{0}$ is known to the ticket recipient, the ticket recipient can compute the opening as $comm_{n-1} = \mathsf{h}^{n-1}(comm_0)$. On redeeming a ticket, the smart contract verifies that $$\mathsf{Open}(comm_i, comm_{i-1}) = \top$$ and sets $channel.comm[redeemer] \leftarrow comm_{i-1}$. Thus the next time the node redeems a ticket, it must reveal $comm_{i-2}$. In addition, each node is granted the right to reset the commitment to a new value. This is particularly necessary once a node reveals $comm_0$ and therefore is with high probability unable to compute a value $r$ such that $$\mathsf{Open}(comm_0,r) \neq \bot,$$ where $\bot$ represents the truth value ``false". Since this mechanism can be abused by the ticket recipient to tweak the entropy used to determine whether a ticket is a winner or not, the smart contract keeps track of resets of the on-chain commitment and sets $$channel[redeemer].ticketEpoch \leftarrow channel[redeemer].ticketEpoch +1 ,$$ thereby invalidating all previously unredeemed tickets.
