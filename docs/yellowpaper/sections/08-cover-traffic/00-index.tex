\section{Cover Traffic}
\label{sec:ct}

Cover traffic (CT), also known as chaff, is randomly generated packets injected into a mixnet to increase its anonymity set. Since cover traffic is transported and mixed using the same procedures as `real' packets, cover traffic is indistinguishable from real traffic. Deploying cover traffic makes it more difficult for an attacker to conduct passive attacks against the network. This is a direct result of there being more traffic: passive attacks such as traffic analysis attacks become more expensive the more traffic there is to analyse.

Cover traffic also increases the anonymity level of the mix network by improving sender-recipient unlinkability. This is particularly important in the early stages of HOPR's lifecycle, since it can be reasonably expected that traffic through the network will initially be low. Without cover traffic, it would be easier for a global passive adversary to link the sender and recipient of a particular packet.

A drawback of cover traffic is that it increases bandwidth overhead, which ultimately reduces the capacity of the network. This trade-off between capacity, latency, and anonymity is a result of the ``anonymity trilemma" \cite{AnonymityTrilemma}. Since HOPR is a privacy network, it is reasonable to prioritise anonymity here. However, it is important to mitigate the deleterious effects of cover traffic as far as possible, particularly avoiding wasted transmissions. This is achieved through judicious choice of which nodes to open cover traffic channels to and select paths to, based on node importance, and closing covert traffic channels to nodes which are found to be offline. 

\subimport{}{01-cover-traffic-nodes.tex}
\subimport{}{02-channel-open.tex}
\subimport{}{03-path-selection.tex}
\subimport{}{04-channel-closure.tex}