
\begin{abstract}

\end{abstract}
\section{Introduction}
Internet privacy, also commonly referred to as online privacy, is a subset of data privacy and a fundamental human right. We consider privacy to revolve around control, use and disclosure of one’s personally identifiable information.
However, our increasingly technologically-driven world puts great pressure on privacy. 
This is why several actors in the space are continuesly developinng different solutions to achieve anonymous communication and thus leveraging internet privacy.


\subsection*{State of the art}
There are few projects working on solving privacy problems in the blockchain space like Monero, Zcash and Starkware. These projects rely on advanced zero knowledge verifiable computing techniques like Bulletproofs, ZKSnarks  and ZKStarks. These techniques allow a prover to convince a verifier about a certain claim by only sharing a proof which backs up the prover’s claim without sharing any private information. ZKStarks is the most scalable solution between all of them since it does not require a trusted setup phase. The trusted setup scales linearly with the computing time in ZKSnarks. This is why, the key and proof generation are done only once off-chain. 
\\Another project worth mentioning, Enigma which uses a hardware approach Intel SGX to guarantee privacy preserving data computation. This approach has been found to be vulnerable to some attacks (Meltdown, Spectre and Foreshadow) which allow the attacker to access and modify sensitive data. 




