\subsection{Path Selection}
HOPR uses a pseudo-random selection algorithm to determine the identities of nodes participating in the network relaying service.
\\For each round, a subset $k$-of-$n$ nodes will be chosen where $n$ is the total number of nodes in the HOPR network. The selection is divided into two steps:
\begin{enumerate}
    \item Pre-selection: 
    During this phase, a subset $m$-of-$n$ nodes will be selected based on different factors:
    \begin{itemize}
        \item Availability
        \item Payment channel graph
        \item Reputation
        \item Stake
    \end{itemize}
Each node gets a score that is proportional to the previously mentioned factors.
\item Random selection: 
Each edge (from node $a$ to $b$) within the subset $m$ is assigned a random number $r_i$. Edges are then sorted by $r_i*score(edge_i)$
\end{enumerate}
Once an edge is selected, it is added to the current path. All paths are then sorted according to their weight and the path with the highest score is expanded next.


\subsubsection{Availability}
Availability is estimated using the heartbeat protocol.
Each node maintains a list of neighbor nodes in the network and either ping or passively listen to them in order to determine whether they are online or offline.
A node is considered online if the ping response (“PONG”) comes back within a certain timeframe. Otherwise, the node is considered offline and its waiting time for the next PING attempt is doubled.


\subsubsection{Payment channel graph}
Every node that intends to send messages needs to have a basic understanding about the topology of the network which means whether the channel is open and funded with enough HOPR tokens for relaying. In case no existing payment channel is open, the sender creates a new channel and funds it with enough HOPR tokens.

\subsubsection{Reputation} The node’s reputation either increases or decreases its probability of being chosen depending on its behavior. We assume the majority of nodes are honest but there are some who misbehave to attack the network. For that reason, HOPR plans to implement a reputation system (check future work for more information.
\subsubsection{Stake}
The HOPR tokens are used to create payment channels with other nodes in the network and thereby staked in the HOPR network. The node will then use the HOPR token to cover transaction costs when interacting with the blockchain. The more stake a node locks the higher probability that it would be chosen as a relayer.











    
