\subsection{Path Selection}
HOPR uses a pseudo-random selection algorithm to determine the identities of nodes participating in the network relaying service.
\\For each round, a subset $k$-of-$n$ nodes will be chosen. The selection is divided into two steps:
\begin{enumerate}
    \item Pre-selection: 
    During this phase, a subset $m$-of-$n$ nodes will be selected based on different factors:
    \begin{itemize}
        \item Availability
        \item Payment channel graph
        \item Reputation
        \item Stake
    \end{itemize}
Each node gets a score that is proportional to the previously mentioned factors.
\item Random selection: 
Each edge (from node $a$ to $b$) within the subset $m$ is assigned a random number $r_i$. Edges are then sorted by $r_i*score(edge_i)$
\end{enumerate}
Once an edge is selected, it is added to the current path. All paths are then sorted according to their weight and the path with the highest score is expanded next.


\subsubsection{Availability}
Availability is estimated using the heartbeat protocol. A heartbeat is a periodic signal generated by hardware or software to indicate normal operation or to synchronize other parts of a computer system.
\\ Each node maintains a list of neighbor nodes in the network and either ping or passively listen to them in order to determine whether they are online or offline.

\subsubsection{Payment channel graph}
\subsubsection{Reputation}
\subsubsection{Stake}











    
