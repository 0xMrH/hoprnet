\subsection{Sphinx Packet Format}

A sphinx packet consists of two parts:
\begin{enumerate}
\item Header:
\begin{itemize}
\item Key derivation
\item Routing information
\item Integrity protection
\end{itemize}
\item Body:
\begin{itemize}
\item Onion-Encrypted payload
\end{itemize}
\end{enumerate}
\paragraph{Notation}
Let $k$ be a security parameter. An adversary will have to do about $2^k$ work to break the security of Sphinx with non negligible probability. We suggest using $k=128$.
Let $r$ be the maximum number of nodes that a Sphinx mix message will traverse before being delivered to its destination.
$G$ is a prime order cyclic group satisfying the Decisional Diffie-Hellman Assumption. The element $g$ is a generator of $G$ and $q$ is the (prime) order of $G$, with $q\approx 2^k$.
$G^*$ is the set of non-identity elements of $G$.
$h_b$ is a hash function which we model by random oracles such that:
$h:G^*\times G^*\rightarrow \mathbb{Z}^*_q$ where $\mathbb{Z}^*_q$ is the field of non-identity elements of $\mathbb{Z}_q$ (field of integers).
Each node $n\in \mathbb{N}$ has a private key $x_n\in \mathbb{Z}^*_q$ and a public key $y_n=g^{x_n}\in G^*$ where $\mathbb{N} \subset \{0,1\}^k$is a set of mix nodes identifiers.

\paragraph{Key derivation}
The sender (A) picks a random $x\in \mathbb{Z}^*_q$ that is used to derive new keys for every packet. 
\newline (A) randomly picks a path consisting of intermediate nodes (B), (C),(D) [see section path-finding] and the final destination of the packet (E) 
\newline (A) performs an offline Diffie-Hellman key exchange with each of these nodes and derives shared keys with each of these nodes.
\newline (A) computes a sequence of $r$ tuples (in our case r=4)  $$(a_0,s_0,b_0),.................,(a_{r-1},s_{r-1},b_{r-1})$$ as follows:
\begin{itemize}
\item $a_0=g^x,s_0=y^x_B,b_0=h(a_0,s_0)$
\item $a_1=g^{xb_0},s_1=y^{xb_0}_C,b_1=h(a_1,s_1)$
\item $a_2=g^{xb_0b_1},s_2=y^{xb_0b_1}_D,b_2=h(a_2,s_2)$
\end{itemize}
 Where $y_B,y_C,y_D,y_E$ are the public keys of the nodes $B,C, D$  which we assume are available to $A$ . The $a_i$ are the group elements which, when combined with the nodes’ public keys, allows computing a shared key for each via Diffie-Hellman key exchange, and so the first node in the user-chosen route can forward the packet to the next, and only that mix-node can decrypt it.
The $s_i$ are the Diffie Hellman shared secrets, and the $b_i$ are the blinding factors.


\paragraph{Routing information}
Each node on the path needs to know the next downstream node. Therefore, the sender $(A)$ generates routing information $\beta_i$ for $(B)$, $(C)$ and (D) as well as message END to tell $(E)$ that it is the final receiver of the message. 
\newline As $(A)$ has a shared secret with each of the nodes along the path, it is able to derive blindings for each of them which is symbolised as different hatchings.
\newline Once $(B)$ receives the packet, it derives the shared key $s_0$(for simplicity we call it $s_B$ as it is the shared key with $B$) by computing
$$s_0=(a_0)^b=(g^x)^b=(g^b)^x=y^x_B$$ and removes its blindings. This allows $(B)$ to unblind the routing info that tells $(B)$ the public key of the next downstream node $(C)$.
The unblinding works as follow:
\begin{enumerate}
    \item $B$ computes the keyed-hash of the encrypted routing information $\beta_0$ as $HMAC(s_0,\beta_0)$ and compares with the integrity tag $\gamma_0$ attached in the packet header. If the integrity check fails because the header has been tampered with, the packet is dropped. Otherwise, the mix-node proceeds to step 2.
    \item $B$ is now ready to decrypt the attached $\beta_0$. In order to extract the routing instructions, the mix-node $B$ first appends a zero-byte padding at the end of $\beta_0$ and decrypts the padded block of routing information $B$ by XORing it with $(h(s_0))$. Where $\varrho :\{0,1\}^k\rightarrow \{0,1\}^{(2r+3)k}$ is a pseudo-random generator (PRG) and $h_\varrho:G^*\rightarrow \{0,1\}^k$ is a hash function used to key $\varrho$ . 
    \item $(B)$ parses the routing instructions from $(A)$ in order to obtain the address of the next mix-node $(C)$, as well the new integrity tag $\gamma_1$ and $\beta_1$, which should be forwarded to the next hop.
    \item $(B)$ blinds the key share $a_0=g^x$ by setting it to $a1=g^{xb_0}$.
\end{enumerate}
$(B)$ also removes one layer of encryption from the payload.
The payload $\delta_0$ in the Sphinx packet is computed using a wide-block cipher to ensure that, if an adversary modifies the payload at any point, the message content is irrecoverably lost.
\newline Once $(C)$ receives the packet, it derives the shared key $s_c$ and removes the blinding, extracts the public key of $(D)$ and deletes the routing information from the packet. Afterwards, it fills the empty space with its own blinding which is different from the one of $(B)$. 
\newline Same happens at $(D)$: key derivation, unblinding, deleting, shifting, decryption and blinding.
\newline Last but not least, the packet arrives at $(E)$, the final destination of the packet. Like the other nodes, $(E)$ first derives its shared key $s_E$ and removes the blinding. In contrast to the previous nodes, namely $(B)$, then $(C)$, then $(D)$, it finds a message that symbolizes the end of path and tells $(E)$ that it’s the recipient.









