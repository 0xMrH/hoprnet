\subsection{Sphinx Packet Format}

A sphinx packet consists of two parts:
\begin{enumerate}
\item Header:
\begin{itemize}
\item Key derivation
\item Routing information
\item Integrity protection
\end{itemize}
\item Body:
\begin{itemize}
\item Onion-Encrypted payload
\end{itemize}
\end{enumerate}
\paragraph{Notation}
Let $k$ be a security parameter. An adversary will have to do about $2^k$ work to break the security of Sphinx with non negligible probability. We suggest using $k=128$.
Let $r$ be the maximum number of nodes that a Sphinx mix message will traverse before being delivered to its destination.
$G$ is a prime order cyclic group satisfying the Decisional Diffie-Hellman Assumption. The element $g$ is a generator of $G$ and $q$ is the (prime) order of $G$, with $q\approx 2^k$.
$G^*$ is the set of non-identity elements of $G$.
$h_b$ is a hash function which we model by random oracles such that:



